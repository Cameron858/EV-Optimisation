\documentclass{article}
\usepackage{amsmath}
\usepackage{amsfonts}
\usepackage{amssymb}
\usepackage{graphicx}
\usepackage[a4paper,top=2cm,bottom=2cm,left=3cm,right=3cm,marginparwidth=1.75cm]{geometry}
\usepackage{booktabs}

\title{EV Optimisation: Physics Model Overview}
\date{}

\begin{document}

\maketitle

This document provides a brief overview of the physics model implemented in the \\* \textbf{EV-Optimisation} project \cite{EVOptimisationRepo}, a Python-based tool for optimising electric vehicle performance.

\section{Core Conversions}

Basic unit conversion functions are used throughout the model:
\begin{itemize}
    \item \textbf{RPM to radians/second:}
    $\text{rads} = (\text{rpm} \times 2\pi) / 60$
    \item \textbf{km/h to m/s:}
    $\text{m/s} = \text{km/h} / 3.6$
    \item \textbf{m/s to km/h:}
    $\text{km/h} = \text{m/s} \times 3.6$
\end{itemize}

\section{Forces and Resistance}

The model accounts for various forces acting on the vehicle:

\subsection{Motor Driving Force}
The driving force at the wheels ($F$) generated by the motor is calculated using:
$$F = \frac{P}{w_w r_t}$$
where $P$ is motor power, $w_w$ is wheel angular velocity, and $r_t$ is tire radius. The wheel angular velocity is derived from the motor's angular velocity and the gear ratio: $w_w = w_m / \text{gear\_ratio}$.

\subsection{Rolling Resistance}
The coefficient of rolling resistance ($c_r$) is determined by tire pressure and velocity, based on data from Engineering Toolbox \cite{EngToolboxRollingResistance}:
$$c_r = 0.005 + \left(\frac{1}{\text{tire\_pressure}}\right) \times \left(0.01 + 0.0095 \times \left(\frac{\text{velocity}}{100}\right)^2\right)$$
The rolling resistance force ($F_{rr}$) is then:
$$F_{rr} = c_r \times m \times g$$
where $m$ is the vehicle mass and $g$ is the acceleration due to gravity (9.81 m/s$^2$).

\subsection{Drag Force}
The drag force ($F_d$) due to air resistance is calculated as:
$$F_d = c_d \times 0.5 \times \rho \times v^2 \times A$$
where $c_d$ is the drag coefficient, $\rho$ is the air density (1.2 kg/m$^3$ at NTP), $v$ is the velocity, and $A$ is the frontal area.

\section{Performance Metrics}

\subsection{Time to Battery Drain (Range Calculation)}
For range calculation, the time for the battery to drain at a constant speed is computed.
1. Power required $P_W = F \times v$, where $F$ is the total resistive force at constant speed and $v$ is the cruising speed.
2. Account for drivetrain efficiency ($\eta$): $P_{kW, \text{actual}} = \frac{P_W / 1000}{\eta}$.
3. Time to drain ($t_{hrs}$) is then:
$$t_{hrs} = \frac{\text{battery\_kWh}}{P_{kW, \text{actual}}}$$
The range is then found by multiplying $t_{hrs}$ by the cruising speed.

\subsection*{Time to Target Speed (Acceleration)}
The time required to reach a target speed (e.g., 100 km/h) is calculated using Euler integration. The net force ($F_{net}$) is determined by the constant driving force from the motor minus the resistive forces (rolling and drag).
$$F_{net} = F_{drive} - (F_{drag} + F_{rolling})$$
The acceleration ($a$) is then $a = F_{net} / m$.
The velocity ($v$) and time ($t$) are updated iteratively with a small time step ($\Delta t$):
$$v_{new} = v_{old} + a \times \Delta t$$
$$t_{new} = t_{old} + \Delta t$$
This process continues until the target speed is reached.

\section{Default Vehicle Configuration}

The physics model uses a `VehicleConfig` dataclass to define default parameters for simulations. These values can be overridden, but the default settings are as follows:

\begin{table}[h!]
    \centering
    \begin{tabular}{lc}
        \toprule
        \textbf{Description} & \textbf{Default Value} \\
        \midrule
        Tire pressure & 2.5 bar \\
        Motor max RPM & 6000 RPM \\
        Tire radius & 0.65 m \\
        Frontal area & 2.2 m$^2$ \\
        Drag coefficient & 0.25 \\
        Gear ratio & 10 (unitless) \\
        Cruising speed & 100 km/h \\
        Drivetrain efficiency & 1.0 (0-1) \\
        \bottomrule
    \end{tabular}
    \caption{Default Vehicle Configuration Parameters}
    \label{tab:vehicle_config}
\end{table}

\bibliographystyle{plain}
\bibliography{references}

\end{document}